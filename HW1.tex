\documentclass[11pt, oneside]{article}   	% use "amsart" instead of "article" for AMSLaTeX format
\usepackage{geometry}                		% See geometry.pdf to learn the layout options. There are lots.
\geometry{letterpaper}                   		% ... or a4paper or a5paper or ... 
%\geometry{landscape}                		% Activate for rotated page geometry
\usepackage[parfill]{parskip}    		% Activate to begin paragraphs with an empty line rather than an indent
\usepackage{graphicx}				% Use pdf, png, jpg, or eps§ with pdflatex; use eps in DVI mode
								% TeX will automatically convert eps --> pdf in pdflatex		
\usepackage{amssymb}
\usepackage{tcolorbox}
\usepackage{url}

\usepackage{amsmath}
%SetFonts

%SetFonts


\title{Homework 1}
\author{CS 4364/5364\\Spring 2021}
\date{Due: 10 February 2021}							% Activate to display a given date or no date

\begin{document}
\maketitle

Because of the reliance of the particular assignments in this class on mathematical notation, 
and the fact that all assignments will be submitted electronically, 
students are encouraged to use \LaTeX{} to formalize their responses. 
\textbf{For those enrolled in the graduate section the use of latex is \emph{required}.}
This assignment (like all others) will be posted on the course \texttt{github}\footnote{\url{github.com/deblasiolab/CS4364-documents}} as source code as well as in PDF form on the course website. 
Please submit your assignment to the professor via email, either as a link to your assignment online (i.e. overleaf or github) or as an attachment. 
Graduate students will need to include the \texttt{.tex} files as well as a PDF, this is optional but encouraged for undergraduates. 

\begin{enumerate}
\item \textbf{(20 points)} 
We talked in class about writing the knapsack problem as an integer linear program. 
Remember for the knapsack problem (without replacement) you are given the following:
\begin{itemize}
\item $w_{i}$ -- the weight of item $i$ for $i = 1 \dots n$, 
\item $v_{i}$ -- the value of each item $i$ for $i = 1 \dots n$, and 
\item $m$ -- the maximum weight you can cary in your knapsack. 
\end{itemize}

The goal is to find a set of items with maximum value such that the total weight of the items is less than or equal to the maximum weight you can cary. 

Your task for this homework assignment is to formalize the ILP for the 0/1 knapsack problem, 
and provide an explanation to how each component (each constraint or group of constraints and the objective equation) 
contributes to solving the problem. 
The definition should be in the same form as the example below. 
The explanation should be in paragraph form following the solution. 
The \emph{variables} in your ILP will be the set $x_{i}$ of binary values (i.e. $x_{i} \in \{1,0\}$) that states if item $i$ is in ($1$) or not in ($0$) your knapsack.

\begin{tcolorbox}[colback=blue!5,colframe=blue!75!black,title=Writing down an ILP] 
In this example below we are minimizing our objective function, our variables are $x_{1}, x_{2}, \& x_{3}$, and we have a collection of maximum values $m_{1}, m_{2}, \& m_{3}$. 
As a reminder the objective function (which is being maximized or minimized) defines what makes a solution \emph{optimal}, 
and the set of constraints (linear inequalities) define what makes a solution \emph{valid}
As an aside, even though the values of $m_{x}$ are arbitrary that are not technically variables that we're trying to optimize. 
\begin{equation*}
\begin{array}{ll@{}ll}
\text{minimize}  & 7 x_{1} + 14 x_{2} - 12 x_{3} \\
\text{subject to}& x_{i} \leq m_{i}  &1 \le i \le 3\\
			& \displaystyle\sum_{1 \le i \le 3}  x_{i} \geq 4,  &\\
			& x_{i} \in \mathbb{Z}, &1 \le i \le 3
\end{array}
\end{equation*}
(This ILP has no grounding in a real-world problem, I made it up.) 

Note that the first and last line of the constraints section actually defines a collection of inequalities, one for each value of $i$. 
\end{tcolorbox}

\item \textbf{(20 points)} 
Give an algorithm that takes in two string $\alpha$ and $\beta$, of length $n$ and $m$, and finds the longest suffix of $\alpha$ that exactly matches a prefix of $\beta$. 
The algorithm should run in $O(n+m)$ time. 
(Hint: your algorithm will likely rely on the fact that the $Z_{i}(S)$ values can be computed in time thats linear with respect to the length of a string $S$). 

Your response should not only include a (high level) description of the algorithm in english prose (with additional pseudocode if absolutely necessary) 
but a description of: (1) why the algorithm defined solved the problem and does not leave any corner cases, 
as well as (2) an analysis of the running time to ensure that it fulfills the questions requirements. 
\end{enumerate}


\end{document}  