\documentclass[11pt, oneside]{article}   	% use "amsart" instead of "article" for AMSLaTeX format
\usepackage{geometry}                		% See geometry.pdf to learn the layout options. There are lots.
\geometry{letterpaper}                   		% ... or a4paper or a5paper or ... 
%\geometry{landscape}                		% Activate for rotated page geometry
%\usepackage[parfill]{parskip}    		% Activate to begin paragraphs with an empty line rather than an indent
\usepackage{graphicx}				% Use pdf, png, jpg, or eps§ with pdflatex; use eps in DVI mode
								% TeX will automatically convert eps --> pdf in pdflatex		
\usepackage{amssymb,amsthm}
\usepackage[table]{xcolor}

%SetFonts

%SetFonts
\usepackage{amsmath}
\usepackage{MnSymbol}

\DeclareMathOperator*{\argmax}{argmax}

\title{Homework 2 Solution}
\author{CS 4364/5364\\Spring 2021}
\date{Due: 25 February 2021}									% Activate to display a given date or no date

\begin{document}
\maketitle

This homework is worth 5 points out of the total 25 points of homework in the class. 

\begin{enumerate}


\item
\textit{Given two sequences $S$ and $T$ (not necessarily the same length), 
let $G$, $L$, and $H$ be the scores of an optimal global alignment, an optimal local alignment, 
and an optimal global alignment without counting the indels at the beginning of $S$ and the end of $T$, respectively. 
\begin{enumerate}
\item Give an example of $S$ and $T$ so that all 3 scores, $G$, $L$, and $H$, are different. 
\item Prove or disprove the statement $L \ge H \ge G$.
\end{enumerate}
}

\textbf{part (a)}
Let $A \in \Sigma^n$ be an arbitrary string of length $n\ge1$ over some alphabet, 
and let $x,y,z,w \notin \Sigma$ be 4 characters not contained in $A$. 
Define $S=\{x\}^k\cdot A \cdot \{y\}^\ell$ and $T=\{z\}^\ell\cdot A \cdot \{w\}^k$, 
where $\cdot$ is the concatination operator, $\{a\}^i$ is a string consisting of the letter $a$ repeated $i$ times, 
and $k > \ell$. 

The best local alignment will align the substrings $A$ in both $S$ and $T$ to each other, any extension would decrease the score (assuming the indel and mismatch values would penalize those operations). 
The best semi-global alignment would add $2\ell$ mismatches or indels (whichever provides a lower penalty) to the local alignment since the beginning of $T$ and the end of $S$ must be aligned and those characters cannot be matches. 
The best global alignment would have an additional $2(k-\ell)$ mismatches or $2k$ indels (again depending on the penalties). 

\textbf{part (b)}
Let $f(\mathbb{A})$ be the alignment score for alignment $\mathbb{A}$. 
And let $G(A,B)$ be the optimal global alignment of sequences $A$ and $B$. 
In the local sequence alignment algorithm we are searching over the set of all substrings, 
so we can actually restate that as searching for the best alignment out of the set 
\[
 \mathcal{A}_\ell =: \left\{G\left(S[i...j],T[k...\ell]\right) \mid 1 \le i , j \le |S|, 1 \le k , \ell \le |T|\right\} \cup \{\mathbb{A}_\emptyset\},
\]
where $\mathbb{A}_\emptyset$ is the alignment of two empty strings. 
We can rewrite the definition of local alignment as:
$\mathbb{A}_\ell =: \argmax_{\mathbb{A} \in \mathcal{A}_\ell} \left\{ f\left(\mathbb{A}\right)  \right\}$. 

Similarly, the set of alignments being considered by semi-global alignment (as defined in the problem) is 
\[
 \mathcal{A}_s =: \left\{G\left(S\left[i...\left|S\right|\right],T\left[1...\ell\right]\right) \mid 1 \le i \le |S|, 1 \le \ell \le |T|\right\}  \cup \{\mathbb{A}_\emptyset\}
\]
and the semi-global alignment problem reduces to $\mathbb{A}_s =: \argmax_{\mathbb{A} \in \mathcal{A}_s} \left\{ f\left(\mathbb{A}\right)  \right\}$. 
Finally let the global alignment be $\mathbb{A}_g =: G(S,T)$ and thus the set considered by the problem contains only one element  $ \mathcal{A}_g =: \left\{\mathbb{A}_g\right\}$.

As defined $ \mathcal{A}_g \subset \mathcal{A}_s \subset \mathcal{A}_\ell$. 
We know that $f(\mathbb{A}_g) \le f(\mathbb{A}_s)$ since $\mathbb{A}_g \in \mathcal{A}_s$, and $ f(\mathbb{A}_s) \ge f(\mathbb{A}')$ for all $\mathbb{A}' \in \mathcal{A}_s$.
The same argument applies to the other cases. 
\qed

%\item \textbf{(1 Point)}
%Under the affine gap model, describe an algorithm that find the optimal \textit{local} alignment of strings $S[1...n]$ and $T[1...m]$ in $O(mn)$-time. 

\item 
\textit{Given two strings $S[1...n]$ and $T[1...m]$, we would like to find the two non-overlapping alignments 
($S[i_1...i_2],T[j_1...j_2]$) and ($S[i_3...i_4],T[j_3...j_4]$) such that $i_2<i_3$ and $j_2<j_3$ and the total alignment score is maximized 
in running time $O(mn)$. 
Hints: remember that the local alignment between two sequences is the alignment of a pair of substrings from $S$ and $T$ such that the alignment score is maximized, 
and that the question does not say anything about the relationship between $i_1$ and $i_2$ nor $i_3$ and $i_4$ that if $i_3>i_4, S[i_3...i_4]$ is the empty string.}

\textbf{Algorithm}
\begin{enumerate}
\item using the recurrence for local alignment, create the matrix $V$, and a maximum value array $M$, where 
\[
M(i,j) =: \max \begin{cases}
M(i-1,j-1)\\
M(i-1,j)\\
M(i,j-1)\\
V(i,j)\\
\end{cases}.
\]
The $M$ array keeps track of the maximum value local alignment between $S[1...i]$ and $T[1...j]$. \label{step:forwardLocal}
\item using the same procedure as Step~\ref{step:forwardLocal} create $V'$ and $M'$ for the reverse of $S$ and $T$. \label{step:reverseLocal}
\item perform a linear scan to find 
\[
(\hat{i},\hat{j}) = \argmax_{(i,j): 1 \le i < n, 1 \le j < m} \left\{M(i,j) + M'(i+1,j+1)\right\}
\] \label{step:linearScan}
\item let $(i_2,j_2)$ be the index in $V[i_2][j_2] = M(\hat{i},\hat{j})$ (assuming $M(i,j) \ne 0$, if it is $0$, set $(i_2,j_2) = (1,1)$)\label{step:forwardScan}
\item perform a traceback in $V$ to find $(i_1, j_1)$. (if $V(i_2,j_2) == 0$ set $(i_1,j_1) = (0,0)$ to signify the empty strings.) \label{step:forwardTraceback}
\item let $(i_3, j_3)$ be the index in $V'[i_3][j_3] = M'(\hat{i}+1,\hat{i}+1)$ (assuming $M'(\hat{i}+1,\hat{i}+1) \ne 0$, if it is $0$, set $(i_3,j_3) = (n,m)$).\label{step:reverseScan}
\item perform a traceback in $V'$ to find $(i_4,j_4)$. (if $V(i_3,j_3) == 0$ set $(i_4,j_4) = (n-1, m-1)$ to signify the empty strings.)\label{step:reverseTraceback}
\end{enumerate}

The linear scan in Step~\ref{step:linearScan} finds the boundary between the regions where the two maximal sets of substrings exist. 
Keeping the $M$ array keeps us from having to do a quadratic time scan for each possible partition of the local alignment space. 
We can then find the maximal substrings in each of the regions using the techniques we already know.
It is possible for one of the two substrings to be empty, in this case either $M(i,j) = 0$ or $M'(i+1,j+1)=0$, 
in that case the whole maximal set of substrings would be in one partition, and nothing positive would be in the other. 

Filling in $V, M, V', $ and $M'$ in Steps~\ref{step:forwardLocal} \&~\ref{step:reverseLocal} takes $O(mn)$-time each. 
The linear scan in Step~\ref{step:linearScan} takes $O(mn)$-time.
 The two linear scans in Steps~\ref{step:forwardScan} \&~\ref{step:reverseScan} take a total maximum of $O(mn)$-time. 
 Finally, the tracebacks in Steps~\ref{step:forwardTraceback} \&~\ref{step:reverseTraceback} take a total maximum time of $O(mn)$-time. 
This means there is a constant number of $O(mn)$-time steps. \qed

\end{enumerate}

\end{document}  