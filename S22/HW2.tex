\documentclass[11pt, oneside]{article}   	% use "amsart" instead of "article" for AMSLaTeX format
\usepackage{geometry}                		% See geometry.pdf to learn the layout options. There are lots.
\geometry{letterpaper}                   		% ... or a4paper or a5paper or ... 
%\geometry{landscape}                		% Activate for rotated page geometry
\usepackage[parfill]{parskip}    		% Activate to begin paragraphs with an empty line rather than an indent
\usepackage{graphicx}				% Use pdf, png, jpg, or eps§ with pdflatex; use eps in DVI mode
								% TeX will automatically convert eps --> pdf in pdflatex		
\usepackage{amssymb}
\usepackage{tcolorbox}
\usepackage{url}

\usepackage{amsmath}
%SetFonts
%\usepackage{multicolumn}
\usepackage{multirow}
%SetFonts


\title{Homework 2}
\author{CS 4364/5364\\Spring 2022}
\date{Due: 9 February 2022}							% Activate to display a given date or no date

\begin{document}
\maketitle

Because of the reliance of the particular assignments in this class on mathematical notation, 
and the fact that all assignments will be submitted electronically, 
students are encouraged to use \LaTeX{} to formalize their responses. 
\textbf{For those enrolled in the graduate section the use of latex is \emph{required}.}
This assignment (like all others) will be posted on the course \texttt{github}\footnote{\url{github.com/deblasiolab/CS4364-documents}} as source code as well as in PDF form on the course website. 
Please submit your assignment to the professor via email, either as a link to your assignment online (i.e. overleaf or github) or as an attachment. 
Graduate students will need to include the \texttt{.tex} files as well as a PDF, this is optional but encouraged for undergraduates. 

\begin{enumerate}
\item \textbf{(25 points)} 
The first method we saw to find a pattern $P$ in a text $T$ was using the maximum prefix overlap values on a special string (calculate $M_i$ for each $i$ in the string $P$\texttt{\$}$T$).
We know that we can compute these $M_i$ values in $O(m+n)$ time (assuming $|P|=n$ and $|T|=m$).
The solution described in class assumes both strings are over the same alphabet: what if they were not?

Consider the following:
Given a protein sequence pattern $P\in\Sigma_{AA}^*$ over the amino acid alphabet, and a RNA text $T\in\Sigma_{RNA}^*$ 
(see footnote 
\footnote{$\Sigma_{AA} = \{\texttt{A},\texttt{R},\texttt{N},\texttt{D},\texttt{B},\texttt{C},\texttt{E},\texttt{Q},\texttt{Z},\texttt{G},\texttt{H},\texttt{I},\texttt{L},\texttt{K},\texttt{M},\texttt{F},\texttt{P},\texttt{S},\texttt{T},\texttt{W},\texttt{Y},\texttt{V}\}$ and $\Sigma_{RNA} = \{\texttt{A},\texttt{C},\texttt{U},\texttt{G}\}$}). 
Develop an algorithm that uses the maximum prefix overlap method to determine where the pattern $P$ is in the text $T$ (if it exists),
and runs in $O(m+n)$ time. 

Translating from RNA codons (3 nucleotides) to amino acids is done using the standard codon table (can be found in the slides, though its not actually needed for this exercise). 
The problem is that to reverse the translation of an amino acid $p_i \in P$ there are multiple choice of codon that could have produced it, 
thus there is an exponential number of possible un-translations of $P$.
Therefore, we cannot simply enumerate all possible RNA un-translations of the pattern and run the maximum prefix overlap, it would be exponential time which is not feasible.

Hint: you can run the algorithm multiple times, but as long as this number of repetitions is constant it is absorbed by the big-O.

%\begin{tabular}{r|c|cc|cc|cc|cc|c|l}
%\multicolumn{2}{c}{} & \multicolumn{2}{|c}{A} & \multicolumn{2}{|c}{C} & \multicolumn{2}{|c}{U} &  \multicolumn{2}{|c}{G} & \multicolumn{2}{c}{} \\
%\hline
%\multirow{12}{*}{First} & \multirow{4}{*}{A} &	AAC & \multirow{2}{*}{N} & 	ACC	& \multirow{4}{*}{T} & 	AUC	& \multirow{4}{*}{I} & 	AGC	& \multirow{2}{*}{S} & \multirow{4}{*}{C} & \multirow{12}{*}{Third}\\
%%		AAU		ACU		AUU		AGU		U	
%%		AAA	K	ACA		AUA		AGA	R	A	
%%		AAG		ACG		AUG	M/start	AGG		G	
%%	C	CAC	H	CCC	P	CUC	L	CGC	R	C	
%%		CAU		CCU		CUU		CGU		U	
%%		CAA	Q	CCA		CUA		CGA		A	
%%		CAG		CCG		CUG		CGG		G	
%%	U	UAC	Y	UCC	S	"UUC
%%F	UGC	C	C	
%%		UAU		UCU		UUU		UGU		U	
%%		UAA	stop	UCA		UUA	L	UGA	stop	A	
%%		UAG		UCG		UUG		UGG	W	G	
%%	G	GAC	D	GCC	A	GUC	V	GGC	G	C	
%%		GAU		GCU		GUU		GGU		U	
%%		GAA	E	GCA		GUA		GGA		A	
%%		GAG		GCG		GUG		GGG		G	
%\end{tabular}

\item \textbf{(20 points)}
Given a directed graph $G = (V,E)$, source and sink vertexes $s,t\in V$, and a limit on the flow allowed through nodes $m_e$ (defined $\forall e\in E$).
Write the max flow problem as a linear program, and explain each (set of) equation in your definition. 
Remember the max flow problem assigns a flow (weight), $f_e$, to each \emph{edge} in a graph
while maximizing the total flow going from the source to the sink. 
For each node (other than the source and sink, i.e. $v\in V\setminus\{s,t\}$) 
the total in flow must equal the out flow. 
Some helpful notation to use:
\begin{itemize}
\item a directed edge $e_{(ab)}\in E$ goes from $a$ to $b$
\item the set of \emph{in} edges to a vertex $v$  can be written as $E_{(*v)} = \left\{e_{(a,b)} | b = v,  e_{(ab)}\in E \right\}$
\item the set of \emph{out} edges to a vertex $v$  can be written as $E_{(v*)} = \left\{e_{(a,b)} | a = v, e_{(ab)}\in E\right\}$ 
\item we can say the sum of the in flow to a node $v$ is $\displaystyle\sum_{e \in E_{(v*)}} f_e$
\end{itemize}
Your variables will be the set of $f_e$'s, some of these (but not all) will end up in the optimization function. 
The things to keep in mind that must be satisfied are: 
(1) conservation of flow across nodes, 
(2) maximum flow across an edge, and 
(3) the outflow at the source should equal the inflow at the sink (though this may not need to be explicit in your program). 

\begin{tcolorbox}[colback=blue!5,colframe=blue!75!black,title=Writing down an LP] 
In this example below we are minimizing our objective function, our variables are $x_{1}, x_{2}, \& x_{3}$, and we have a collection of maximum values $m_{1}, m_{2}, \& m_{3}$. 
As a reminder the objective function (which is being maximized or minimized) defines what makes a solution \emph{optimal}, 
and the set of constraints (linear inequalities) define what makes a solution \emph{valid}
As an aside, even though the values of $m_{x}$ are arbitrary that are not technically variables that we're trying to optimize. 
\begin{equation*}
\begin{array}{ll@{}ll}
\text{minimize}  & 7 x_{1} + 14 x_{2} - 12 x_{3} \\
\text{subject to}& x_{i} \leq m_{i}  &1 \le i \le 3\\
			& \displaystyle\sum_{1 \le i \le 3}  x_{i} \geq 4,  &
\end{array}
\end{equation*}
(This LP has no grounding in a real-world problem, I made it up.) 

Note that the first line of the constraints section actually defines a collection of inequalities, one for each value of $i$. 
\end{tcolorbox}

\end{enumerate}
\end{document}