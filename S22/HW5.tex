\documentclass[11pt, oneside]{article}   	% use "amsart" instead of "article" for AMSLaTeX format
\usepackage[margin=1in]{geometry}                		% See geometry.pdf to learn the layout options. There are lots.
\geometry{letterpaper}                   		% ... or a4paper or a5paper or ... 
%\geometry{landscape}                		% Activate for rotated page geometry
\usepackage[parfill]{parskip}    		% Activate to begin paragraphs with an empty line rather than an indent
\usepackage{graphicx}				% Use pdf, png, jpg, or eps§ with pdflatex; use eps in DVI mode
								% TeX will automatically convert eps --> pdf in pdflatex		
\usepackage{amssymb}
\usepackage{tcolorbox}
\usepackage{url}

\usepackage{amsmath}
%SetFonts
%\usepackage{multicolumn}
\usepackage{multirow}
%SetFonts


\title{Homework 5}
\author{CS 4364/5364\\Spring 2022}
\date{Due: 30 March 2022}							% Activate to display a given date or no date

\begin{document}
\maketitle

\textbf{(10 points)} This homework is not technical, therefore students do not ned to submit their responses using \LaTeX, 
but they should submit a PDF or plain text file (\texttt{.txt}, not a Word Document or other file types). 
%Because of the reliance of the particular assignments in this class on mathematical notation, 
%and the fact that all assignments will be submitted electronically, 
%students are encouraged to use \LaTeX{} to formalize their responses. 
%\textbf{For those enrolled in the graduate section the use of latex is \emph{required}.}
%This assignment (like all others) will be posted on the course \texttt{github}\footnote{\url{github.com/deblasiolab/CS4364-documents}} as source code as well as in PDF form on the course website. 
%Please submit your assignment to the professor via email, either as a link to your assignment online (i.e. overleaf or github) or as an attachment. 
%Graduate students will need to include the \texttt{.tex} files as well as a PDF, this is optional but encouraged for undergraduates. 

We want to start thinking about the final project, and begin the process of selecting a topic for discussion.
To help with this, please answer the following questions (note there are no wrong answers):

\begin{enumerate}
\item What topic (or topics) from the foundational concepts (first half of the class) have you found to be most interesting? 

\item What is it about this (these) topic(s) make it (them) stand out? 

\item Is there an aspect of the topic(s) you want to know more about? 

\item (optional) Using the list of computational biology topics\footnote{https://wp1.openzim.org/\#/project/Computational\%20Biology} from the Computational Biology Wikipedia Project as a guide, 
is are there any projects related to your answers above that you are considering? 

\end{enumerate}

\end{document} 
