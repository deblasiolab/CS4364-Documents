\documentclass[11pt, oneside]{article}   	% use "amsart" instead of "article" for AMSLaTeX format
\usepackage[margin=1in]{geometry}                		% See geometry.pdf to learn the layout options. There are lots.
\geometry{letterpaper}                   		% ... or a4paper or a5paper or ... 
%\geometry{landscape}                		% Activate for rotated page geometry
\usepackage[parfill]{parskip}    		% Activate to begin paragraphs with an empty line rather than an indent
\usepackage{graphicx}				% Use pdf, png, jpg, or eps§ with pdflatex; use eps in DVI mode
								% TeX will automatically convert eps --> pdf in pdflatex		
\usepackage{amssymb}
\usepackage{tcolorbox}
\usepackage{url}

\usepackage{amsmath}
%SetFonts
%\usepackage{multicolumn}
\usepackage{multirow}
%SetFonts


\title{Homework 6}
\author{CS 4364/5364\\Spring 2022}
\date{Due: 6 April 2022}							% Activate to display a given date or no date

\begin{document}
\maketitle

Because of the reliance of the particular assignments in this class on mathematical notation, 
and the fact that all assignments will be submitted electronically, 
students are encouraged to use \LaTeX{} to formalize their responses. 
\textbf{For those enrolled in the graduate section the use of latex is \emph{required}.}
This assignment (like all others) will be posted on the course \texttt{github}\footnote{\url{github.com/deblasiolab/CS4364-documents}} as source code as well as in PDF form on the course website. 
%Please submit your assignment to the professor via email, either as a link to your assignment online (i.e. overleaf or github) or as an attachment. 
Graduate students will need to include the \texttt{.tex} files as well as a PDF, this is optional but encouraged for undergraduates. 

\textbf{Question (25 points):} 
Give an algorithm, a proof of correctness, and proof of running time to solve the following problem in $O(kn^2c)$-time.

Given integers $k\ge1$ and $c\ge1$, as well as two strings $S$ and $T$ both of length $n \gg k$, determine a similarity between them that is calculated as the sum of: 
\begin{itemize}
\item $c+1$ times the number of shared distinct exact $k$-mers
\item $c$ times the number of shared distinct $k$-mers with one change that have not been accounted for already
\item $c-1$ times the number of shared distinct $k$-mers with two changes that have not been accounted for already
\item ...
\item $1$ times the number of shared distinct $k$-mers with $c$ changes that have not been accounted for already
\end{itemize}
normalized by the total number of distinct $k$-mers in both sequences.
Here changes are replacements only and not indels. 

As an example for the values of $k=3, c=2$ and the strings \texttt{AACTGT} and \texttt{TGTAAA} the similarity is $\frac{3}{4}$. 
The distinct $3$-mers from the two strings are \texttt{AAC,ACT,CTG,TGT} and \texttt{TGT,GTA,TAA,AAA}.
There is one $3$-mer that matches exactly (\texttt{TGT}).
There is one pair of $3$-mers that match with one change (\texttt{AAC} and \texttt{AAA}).
There is one pair of $3$-mers that match with two changes (\texttt{CTG} and \texttt{GTA})
The remaining pair would need 3 changes (i.e. more than $c$ changes). 
Thus the similarity is 
\[
\frac{(1\times 3) + (1 \times 2) + (1 \times 1)}{8} = \frac{3}{4}.
\]
\end{document} 


