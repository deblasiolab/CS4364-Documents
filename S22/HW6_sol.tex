\documentclass[11pt, oneside]{article}   	% use "amsart" instead of "article" for AMSLaTeX format
\usepackage[margin=1in]{geometry}                		% See geometry.pdf to learn the layout options. There are lots.
\geometry{letterpaper}                   		% ... or a4paper or a5paper or ... 
%\geometry{landscape}                		% Activate for rotated page geometry
\usepackage[parfill]{parskip}    		% Activate to begin paragraphs with an empty line rather than an indent
\usepackage{graphicx}				% Use pdf, png, jpg, or eps§ with pdflatex; use eps in DVI mode
								% TeX will automatically convert eps --> pdf in pdflatex		
\usepackage{amssymb}
\usepackage{tcolorbox}
\usepackage{url}
\usepackage{xspace}

\usepackage{amsmath}
%SetFonts
%\usepackage{multicolumn}
\usepackage{multirow}
%SetFonts

\newcommand{\kmer}{$k$-mer\xspace}
\newcommand{\kmers}{\kmer{}s\xspace}

\title{Homework 6}
\author{CS 4364/5364\\Spring 2022}
\date{Due: 6 April 2022}							% Activate to display a given date or no date

\begin{document}
\maketitle

Because of the reliance of the particular assignments in this class on mathematical notation, 
and the fact that all assignments will be submitted electronically, 
students are encouraged to use \LaTeX{} to formalize their responses. 
\textbf{For those enrolled in the graduate section the use of latex is \emph{required}.}
This assignment (like all others) will be posted on the course \texttt{github}\footnote{\url{github.com/deblasiolab/CS4364-documents}} as source code as well as in PDF form on the course website. 
%Please submit your assignment to the professor via email, either as a link to your assignment online (i.e. overleaf or github) or as an attachment. 
Graduate students will need to include the \texttt{.tex} files as well as a PDF, this is optional but encouraged for undergraduates. 

\textbf{
\textit{Question (35 points):} 
Give an algorithm, a proof of correctness, and proof of running time to solve the following problem in $O(kn^2c)$-time.}

\textbf{
Given integers $k\ge1$ and $c\ge1$, as well as two strings $S$ and $T$ both of length $n \gg k$, determine a similarity between them that is calculated as the sum of: 
\begin{itemize}
\item $c+1$ times the number of shared distinct exact $k$-mers
\item $c$ times the maximum number of shared distinct $k$-mers with one change that have not been accounted for already
\item $c-1$ times the maximumnumber of shared distinct $k$-mers with two changes that have not been accounted for already
\item ...
\item $1$ times the maximum number of shared distinct $k$-mers with $c$ changes that have not been accounted for already
\end{itemize}
normalized by the total number of distinct $k$-mers in both sequences.
Here changes are replacements only and not indels. }

\textbf{As an example for the values of $k=3, c=2$ and the strings \texttt{AACTGT} and \texttt{TGTAAA} the similarity is $\frac{3}{4}$. 
The distinct $3$-mers from the two strings are \texttt{AAC,ACT,CTG,TGT} and \texttt{TGT,GTA,TAA,AAA}.
There is one $3$-mer that matches exactly (\texttt{TGT}).
There is one pair of $3$-mers that match with one change (\texttt{AAC} and \texttt{AAA}).
There is one pair of $3$-mers that match with two changes (\texttt{CTG} and \texttt{GTA})
The remaining pair would need 3 changes (i.e. more than $c$ changes). 
Thus the similarity is 
\[
\frac{(1\times 3) + (1 \times 2) + (1 \times 1)}{8} = \frac{3}{4}.
\]
}

{\Large Algorithm}\\
\begin{enumerate}
\item \label{step:start} Let $\mathcal{S}$ be a set, that will eventually contain all of the distinct \kmers from $S$.
\item \label{step:init} Define the current \kmer as $C_S = s_1s_2...s_k$ and a counter $c_S = k$
\item \label{step:add} If $C_S \notin \mathcal{A}$, add it ($\mathcal{A} = \mathcal{A} \cup \{C\}$).
\item \label{step:next} Update $C_S$ and $c$ to the next \kmer ($c_S++ ; C_S = C_S[2...k]s_{c_S}$).
\item \label{step:gotokkmer} If $c_S \le n$, go to Step~\ref{step:add}. 
\item \label{step:finishExtract}Follow steps above to similarly construct $\mathcal{T}$ from $T$.
\item \label{step:Tkmer} Let the total number of \kmers in both sets be $tk = |\mathcal{S}| + |\mathcal{T}|$.
\item Let the current allowable \kmer difference $d = 0$. 
\item Let the final return value $r = 0$. 
\item Create an empty graph $G$. 
\item \label{step:graph} Create 2 1-dimensional boolean arrays $U_S$ and $U_T$ of length $|\mathcal{S}|$ and $|\mathcal{T}|$ respectively.
\item \label{step:hamming} $\forall s \in \mathcal{S}, t \in \mathcal{T}$, calculate the hamming distance between $s$ and $t$, if it is equal to $d$:
\begin{itemize}
\item Add $s$ and $t$ as vertices to $G$ if $U_S[s]$ and $U_T[t]$ are false respectively.
\item Add the edge $(s,t)$ to $G$.
\item Set $U_S[s]$ and $U_T[t]$ to true. 
\end{itemize}
\item \label{step:mcm} Solve the maximal-cardinality matching problem on $G$. 
\item \label{step:processMatch} For each pair of matched vertices $s$ and $t$:
\begin{itemize}
\item Remove the \kmers from the sets ($\mathcal{S} = \mathcal{S} \setminus \{s\}, \mathcal{T} = \mathcal{T} \setminus \{t\}$). 
\item Update the return $r += (c - d + 1)$.
\end{itemize}
\item \label{step:endDLoop} Update $d = d+1$.
\item \label{step:stop} If $d\le c$ go to Step~\ref{step:graph}.
%\item $\forall s \in \mathcal{S}, t \in \mathcal{T}$, calculate the hamming distance between $s$ and $t$, if it is equal to $d$:
%\begin{itemize}
%\item Remove the strings from the sets ($\mathcal{S} = \mathcal{S} \setminus \{s\}, \mathcal{T} = \mathcal{T} \setminus \{t\}$). 
%\item Update the return $r += (c - d + 1)$.
%\end{itemize}
\item Return $r/tk$.
\end{enumerate}


{\Large Correctness}\\ 
Steps~\ref{step:start} to \ref{step:finishExtract} construct the list of unique \kmers from the strings. 
Since $\mathcal{S}$ and $\mathcal{T}$ are unordered sets, they will be unique in the sets of \kmers they contain. 
By using a known maximal-cardinality matching to find the subset of \kmers that provides the highest score we can ensure we will find the most matches at each value of $d$. 
The algorithm will only look for matched \kmers up to hamming distance $c$ because of the check in Step~\ref{step:stop}.
\kmers can only be matched once because in order for them to be matched they have to first be a node in a graph at some iteration, then be in a matching. 
If this is true, then by Step~\ref{step:processMatch} they will be removed from $\mathcal{S}$ ($\mathcal{T}$) and thus cannot be in a graph is later iterations. 
Also, because each vertex in the graph cannot be in more than one matching a \kmer cannot be matched with more than one other in a single iteration.  

{\Large Running Time}\\
Step~\ref{step:init} takes $O(k)$-time. 
Step~\ref{step:add} takes $O(|\mathcal{S}|)$ time each iteration, and because after Step~\ref{step:next} this increases by 1, and it can be repeated $n$ times from Step~\ref{step:gotokkmer}, the total running time of these 3 steps is $O(n^2)$.
Similarly Step~\ref{step:Tkmer} is also $O(n^2)$. 
Step~\ref{step:graph} can take at most $O(n)$ time, tough the two sets are constantly decreasing. 
Calculating the hamming distance in Step~\ref{step:hamming} will take $O(k)$ time for each pair of \kmers, of which there are at most $n^2$, therefore this step is $O(kn^2)$.
Because we keep a boolean array for each \kmer we don't need to search to determine if its already in the graph, thus the rest of Step~\ref{step:hamming} will take constant time for each possible \kmer pair. 
We know that in a graph with at most $2\times n$ vertices, and $n^2$ edges the running time of a maximum cardinality matching algorithm such as Hopcroft-Karp will be $O(n \sqrt{n^2}) = O(n^2)$.
Step~\ref{step:processMatch} will deal with at most $n$ matches, and processing each one will take constant time to deal with. 
Finally, because of Step~\ref{step:stop} we will run Steps~\ref{step:graph} to~\ref{step:endDLoop} st most $c$ times; 
since Step~\ref{step:hamming} dominates the running time of all of these, 
we see the total running time of the loop is $O(ckn^2)$ time which dominates all of the steps before the loop. 


\end{document} 






