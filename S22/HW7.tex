\documentclass[11pt, oneside]{article}   	% use "amsart" instead of "article" for AMSLaTeX format
\usepackage[margin=1in]{geometry}                		% See geometry.pdf to learn the layout options. There are lots.
\geometry{letterpaper}                   		% ... or a4paper or a5paper or ... 
%\geometry{landscape}                		% Activate for rotated page geometry
\usepackage[parfill]{parskip}    		% Activate to begin paragraphs with an empty line rather than an indent
\usepackage{graphicx}				% Use pdf, png, jpg, or eps§ with pdflatex; use eps in DVI mode
								% TeX will automatically convert eps --> pdf in pdflatex		
\usepackage{amssymb}
\usepackage{tcolorbox}
\usepackage{url}

\usepackage{amsmath}
%SetFonts
%\usepackage{multicolumn}
\usepackage{multirow}
%SetFonts


\title{Homework 7}
\author{CS 4364/5364\\Spring 2022}
\date{Due: 18 April 2022}							% Activate to display a given date or no date

\begin{document}
\maketitle

\textbf{(10 points)}
In the  class' MS Team, under the Wikipedia channel, then files there is a new excel sheet. 
As an individual/team submit the topic you will be using and a link to the page if it already exists. 
Also list all of the team members. 

If the topic you want to do is already on the list contact the person who signed up and have a conversation. 
If you don't know what you want to do but want to work with someone look at the list and contact your peers who have interesting topics. 

(a direct link the the file is \url{https://bit.ly/37j07Do})


\end{document} 
